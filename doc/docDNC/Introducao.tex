
\section{Atenção: Isto é um RASCUNHO da documentação do projeto}

Esse documento é um esboço da documentação do projeto.

Estou desenvolvendo isso como um projeto hobby no pouco tempo livre que tenho, dessa forma, essa documentação estará repleta de erros de portugues, abreviaturas e observações. Além de seções onde eu já pulo de cara para alguma lista de coisas que achei importante registrar logo antes que as idéias se perdessem. Um documento final teria que passar muitas revisões antes de se tornar publicável.

\section{Introdução}


Este documento ainda é apenas um esboço.

Existem pelo menos três formas de dança: individual, em par e em grupo.

Nas danças em par, geralmente, um condutor conduz os movimentos do conduzido. Na cultura tradicional geralmente o condutor é o homem e o conduzido é a mulher. 

Uma das maiores dificuldades enfrentadas é que não existe um sistema simples de representação dos passos de dança acessível à população.

Existe o Syllabus e outras notações de coreografia, mas todos são sistemas muito técnicos e muito complexos.

Em danças em par, por exemplo, o recurso mais utilizado para tentar gravar uma sequencia de passos de dança é o vídeo em que se filmam um par de dançarinos executando os passos da sequencia.

O objetivo é desenvolver um sistema universal, simples, intuitivo de descrição de passos de dança em par de forma que os passos possam ser escritos, armazenados e, posteriormente, reproduzidos.


\subsection{Definições}


\subsubsection{Passo Fundamental}

Passo fundamental é um passo. É o movimento de passo na qual um e apenas um pé se desloca enquanto o outro permanece fixo. O passo pode ser dar de duas formas:

\begin{itemize}
	\item \textbf{passo completo ou passo simples ou simplesmente passo}: Um passo completo é aquele que finaliza com a transferência do peso do corpo para o conjunto perna-pé que deu o passo. O próximo passo que será dado deve ocorrer obrigatoriamente com o pé que permaneceu fixo durante o movimento deste.
	
	\item \textbf{passo para marcação}: Um passo para marcação ou apenas marcação é um passo que não se completa. Ele apenas marca em algum lugar, mas o peso do corpo não é transferido para o mesmo. Dessa forma, o próximo passo que será dado vai ocorrer obrigatoriamente com o mesmo pé que acabou de ser usado para fazer essa marcação. Nas danças em pares, essa marcação pode ser feita apenas tocando algum elemento do pé (ponta ou calcanhar) no chão, mas sem transferir o peso. Nesta linguagem, passos-marcação serão identificados pela presença de um asterisco (*) no código do passo fundamental.
		
\end{itemize}


\subsubsection{Movimento Intermediário ou Caminho}


Um movimento intermediário é uma sequência de passos fundamentais. Por esse motivo, também será chamado de caminho. Um caminho geralmente ocorre dentro de uma unidade de marcação da música. 

Geralmente um caminho começa de um estado inicial e termina em um estado final. E aí será necessário um outro caminho para sair desse estado final e voltar para o primeiro estado inicial. 

Observações:

\begin{itemize}
	\item caminho oposto: Um caminho é oposto a outro caminho quando um desfaz o o que o outro fez no mesmo tempo. O editor de caminhos deveria ser capaz de identificar caminhos opostos. 
\end{itemize}

\subsubsection{Movimento de Dança ou Sequência}

Um movimento de dança é uma sequência de movimentos intermediários ou uma sequencia de caminhos. Nesse trabalho, chamaremos os movimentos de dança com o nome simples de sequencia.


\subsection{Objetivo}

O objetivo é desenvolver uma linguagem de descrição de movimentos de dança em par universal, legível, fácil de ler, entender e escrever. Embora o maior desafio seja descrever as sequencias de  passos, também chamadas de footwork, ou trabalho dos pés, essa linguagem também deve descrever os demais elementos que compõe o movimento como a posição relativa do casal, os estados das mãos e até movimentos de tronco, quadril, pescoço, etc.

Como objetivo secundário, desenvolver ferramentas em linguagem computacional para criar as sequências de movimentos. Um editor de caminhos e um editor de sequências.

É como se eu estivesse desenvolvendo uma linguagem de programação para programar uma sequência de movimentos de dança em par.

Um terceiro e último objetivo, é criar um compilador ou interpretador capaz de ler o programa e interpreta-lo. Qual a saída? Uma dança. Dessa forma, o interpretador é um software capaz de gerar um personagem digital 3D que vai executar os passos de dança na ordem que eles aparecem no programa. 

E assim teremos um sistema completo para escrever sequências de dança e ver os personagens digitais executando esses movimentos.




 


