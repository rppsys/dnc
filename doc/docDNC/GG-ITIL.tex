
\section{ITIL: Conceitos Básicos}

\textbf{O Gerenciamento de Serviços de TI é um conjunto de capacidades organizacionais que são 5:} 
\begin{itemize}
	\item Processos
	\item Métodos
	\item Funções
	\item Papéis
	\item Atividades
\end{itemize}

\textbf{Aí Tio, as CAPACIDADES ORGANIZACIONAIS são 5. Lembrar do computador, a ATIVIDADE de programar ocorre no PAPEL onde você tem FUNÇÕES e MÉTODOS que se tornarão PROCESSOS para prover valor sob a forma de serviços.}

\textbf{As 4 perspectivas (4 Ps) do Gerenciamento de Serviços de TI} 
\begin{itemize}
\item Parceiros - Foco na importância dos fornecedores externos para a entrega do serviço 
\item Pessoas - Foco nas equipes de TI, clientes e demais stakeholder
\item Produtos - Foco em hardware, software, serviços e ferramentas de tecnologia
\item Processos - Foco nos fluxos de atividades necessários à entrega dos serviços
\end{itemize}

% ################################################################

\section{ITIL: Processos}

\textbf{ESTRATÉGIA}
\begin{itemize}
	\item Ger. da Estratégia para Serviços de TI
	\item Ger. de Relacionamento com o Negócio
	\item Ger. de Portfólio de Serviços
	\item Ger. Financeira para Serviços de TI
	\item Ger. de Demandas
\end{itemize}

\textbf{DESENHO}
\begin{itemize}
	\item Coordenação do Desenho
	\item Ger. do Catálogo de Serviços
	\item Ger. de Nível de Serviço
	\item Ger. de Disponibilidade
	\item Ger. de Capacidade
	\item Ger. de Continuidade de Serviços de TI
	\item Ger. de Segurança da Informação
	\item Ger. de Fornecedores
\end{itemize}

\textbf{TRANSIÇÃO}
\begin{itemize}
	\item Ger. de Configuração e Ativos de Serviço
	\item Ger. de Mudanças
	\item Ger. do Conhecimento
	\item Planejamento e Suporte da Transição
	\item Ger. de Liberação e Implantação
	\item Validação e Teste de Serviços
	\item Avaliação de Mudanças
\end{itemize}

\textbf{OPERAÇÃO}
\begin{itemize}
	\item Ger. de Eventos
	\item Ger. de Incidentes
	\item Ger. de Problemas
	\item Cumprimento de Requisições
	\item Ger. de Acesso
	\item Funções: Service Desk, Ger. Técnica,
	\item Ger. de Aplicações, Ger. de Operações
\end{itemize}

\textbf{MELHORIA CONTÍNUA}
\begin{itemize}
	\item Melhoria em 7 passos
\end{itemize}

% ################################################################

\section{ITIL: Livro de Estratégia de Serviços}

\textbf{Os 4 P’s da estratégia de serviços:}
\begin{itemize}
	\item Perspectiva - Direcionamento e visão estratégica Quem são os meus clientes?
	\item Posição - Estratégia de diferenciação Qual o meu posicionamento em relação a cada segmento de Cliente?
	\item Padrão - Características essenciais dos serviços Como dar sustentabilidade para minha estratégia?
	\item Plano - Tradução da estratégia para operaçãoConjunto de ações para realizar minha estratégia?
\end{itemize}


\subsection{Utilidade x Garantia}

\begin{center}
	\emph{Valor para o ITIL é UTILIDADE AND GARANTIA}
	\emph{SLP = Utilidade E Garantia}
\end{center}

\textbf{SLP}
\begin{itemize}
	\item Utilidade gera aumento dos ganhos => Melhoria nos Processos de Negócio
	\item Garantia gera reduçao nas perdas => Variação dos Processos de Negócio
\end{itemize}

\textbf{Utilidade (Adequado ao propósito) OU}
\begin{itemize}
	\item Suporte ao desempenho \textbf{OU} Remoção de barreiras
\end{itemize}

\begin{center}
	\emph{Lembrar de OUtilidade = Desempenho ou Remoção de Barreiras}
\end{center}

\textbf{Garantia (Adequado ao uso) E E E E}
\begin{itemize}
	\item Disponibilidade
	\item Capacidade
	\item Continuidade
	\item Segurança
\end{itemize}

\subsection{Tipos de Provedores de Serviço}

\textbf{Tipo 1 = Interno = Exclusivo para 1 Unidade de Negócio}
\begin{itemize}
	\item Vantagens:  DEDICADO
	\item Desvantagens: DUPLICAÇÃO DE ESFORÇOS
\end{itemize}

\textbf{Tipo 2 = Serviços Compartilhados = Atende a várias unidades de negócio}
\begin{itemize}
	\item Vantagens:  Redução de Custos e Padronização
	\item Desvantagens: Concorrência com o Tipo 3
\end{itemize}

\textbf{Tipo 3 = Externo = Atende a várias organizações}
\begin{itemize}
	\item Vantagens:  Acesso a Melhores Práticas do Mercado, Manutenção do Foco no Negócio
	\item Desvantagens: Depende de terceiros, dificuldade de alcançar vantagens competitivas
\end{itemize}


\subsection{Multi-Sourcing, Multi-Vendor-Sourcing, BPO, KPO e ASP}

\textbf{Diferenciação entre Multi-Sourcing e Multi-Vendor-Sourcing}
\begin{itemize}
	\item Co-Sourcing ou Multi-sourcing = Mistura de fornecedores externos e internos
\end{itemize}

\textbf{Multi-vendor-sourcing = Vários fornecedores externos}

\textbf{Diferenciação entre BPO, KPO e ASP}
\begin{itemize}
	\item BPO, KPO e ASP são modalidades deiferentes de terceirização.
\end{itemize}

\textbf{BPO}
\begin{itemize}
	\item Business Process Outsourcing => Não contrata-se somente a TI, contrata-se o processo de negócio. Mas o processo é mais operacional.
	\item Ex.: Terceirização de Help-Desk
\end{itemize}

\textbf{ASP}
\begin{itemize}
	\item Application Service Provider => Terceiriza-se somente a aplicação ou o software. O processo sou eu que executo dentro de uma aplicação externa.
\end{itemize}

\textbf{KPO}
\begin{itemize}
	\item Knowledge Process Outsourcing => É quando terceirizamos um processo que requer conhecimento específico de negócio. Ex.: Análise de Crédito. Requer um conhecimento especializado. O processo é operacional mas requer um conhecimento mais especializado.
\end{itemize}

\subsection{5 Estágios de Estruturas Organizacionais}
\begin{itemize}
	\item Network (QUASE ANARQUIA) =>
	\item Directive (CENTRALIZA) =>
	\item Delegation (DESCENTRALIZA + PROCESSOS) =>
	\item Coordination (CENTRALIZA + PROCESSOS + HARMONIA) =>
	\item Collaboration (DESCENTRALIZA + PROCESSOS + PARCERIA)
\end{itemize}

\subsection{Processos e Atividades do Volume Estratégia}

\subsubsection{Gerência de Estratégia para Serviços de TI:}

\textbf{Definir como o provedor de serviços pode auxiliar a organização a alcançar seus resultados de negócios.}
\begin{itemize}
	\item Identificar claramente serviços e clientes;
	\item Definir como criar e entregar valor para os clientes;
	\item Identificar oportunidades para fornecer serviços;
	\item Definir formas para explorar as oportunidades;
\end{itemize}

\textbf{Atividades}
\begin{itemize}
	\item Avaliação Estratégica: Análise SWOT
	\item Geração Estratégica: Definição dos 4 P's
	\item Execução Estratégica: Ocorre ao longo do ciclo de vida
\end{itemize}

\subsubsection{Gerência de Portfólio}

\textbf{Garantir composição correta de serviços para atender aos resultados de negócio com um nível adequado de investimento;}
\begin{itemize}
	\item Investigar e decidir sobre quais serviços oferecer;
	\item Acompanhar o investimento em serviços durante o ciclo de vida;
	\item Analisar quais serviços não são mais viáveis e quando eles devem ser descontinuados;
\end{itemize}

\textbf{Categorias de Investimentos} 

\textbf{Transformação}
\begin{itemize}
	\item Investimentos focados em iniciativas em novos espaços de mercado com o desenvolvimento de novas capacidades
\end{itemize}

\textbf{Crescimento}
\begin{itemize}
	\item Investimentos destinados a aumentar o escopo da organização dos serviços
\end{itemize}

\textbf{Execução}
\begin{itemize}
	\item Investimentos centrados em manter as operações
\end{itemize}

\textbf{Conteúdo do Portfólio de Serviços S36}
\begin{itemize}
	\item Composto pelo Pipeline, Catálogo de Serviços e Serviços Descontinuados.
\end{itemize}

\begin{center}
	\emph{Catálogo é o Cardápio.}
	\emph{Portfólio é tudo que o Cheff está aprendendo a fazer, sabe fazer, ou que já fez mas não faz mais.}
\end{center}

\subsubsection{Gerência financeira:}

\textbf{Subprocessos}
\begin{itemize}
	\item Orçamentação (IT operational plan = Budgets)
	\item Contabilização (Cost analysis = Accounting)
	\item Cobrança (Charges)
\end{itemize}

\subsubsection{Gerência de Demandas:}


\textbf{Gerência de Demandas (Estratégia)}
\begin{itemize}
	\item Aqui queremos ENTENDER qual é a demanda, ie, a necessidade do negócio: Fala-se em Business Activity Pattern (BAP) ou Padrão de Atividades de Negócio (PAN)
	\item E aí queremos INFLUENCIAR esse BAP/PAN com base na CAPACIDADE;
\end{itemize}

\textbf{Gerência de Capacidade (Desenho)}
\begin{itemize}
	\item Aqui queremos atender à DEMANDA atual/futura.
\end{itemize}

\subsubsection{Gerência de Relacionamento com o Negócio: É a galera do cafézinho}

\textbf{Estabelecer e manter um relacionamento positivo entre o provedor de serviços e seus clientes;}
\begin{itemize}
	\item Identificar as necessidades dos clientes
	\item Assegurar que o provedor de serviços é capaz de atender estas necessidades com um catálogo de serviços adequado
\end{itemize}

\subsubsection{Abordagem para o Gerenciamento de Riscos}

\begin{center}
	\emph{NÃO É UM PROCESSO}
	\emph{Existe um modelo separado de Gerenciamento de Risco separado só para isso.}
\end{center}

\textbf{Gerenciamento dos riscos}
\begin{itemize}
	\item Implementar respostas
	\item Ganhar confianca sobre a eficacia
	\item Realizar revisao
\end{itemize}

% ##########################################################################

\section{ITIL: Livro Desenho de Serviços}

\begin{center}
	\emph{SLP = U + G(SLR) => Estágio de Desenho => SDP}
\end{center}

\begin{center}
	\emph{Os 5 Aspectos do Desenhos são: Serviços, Porfólio\&Catálogos, Arquiteturas, Processos e Indicadores.}
\end{center}

\subsection{Processos e Atividades do Livro de Desenho de Serviços}

\textbf{DESENHO}
\begin{itemize}
	\item \textbf{Coordenação do Desenho}: ponto único de coordenação e controle
	\item \textbf{Ger. do Catálogo de Serviços}:  fonte única e consistente de informações sobre os serviços acordados
	\item Ger. de \textbf{\emph{Nível de Serviço}}: \emph{Negociar} acordos e \emph{Monitorar} se estão sendo cumpridos
	\item \textbf{Ger. de Disponibilidade} : Garantir que o nível de disponibilidade dos serviços corresponde ou excede as necessidades atuais e futuras do negócio 
	\item \textbf{Ger. de Capacidade} : Garantir capacidade para atender a demanda
	\item \textbf{Ger. de Continuidade de Serviços de TI} 
	\item \textbf{Ger. de Segurança da Informação} 
	\item \textbf{Ger. de Fornecedores} 
\end{itemize}

\begin{center}
	\emph{Ger. Portfólio é do livro de Estratégia, Gew. de Catálogo é do livro de Desenho.}
\end{center}

\begin{center}
	\emph{Catálogo é o Cardápio e Portfólio é tudo que o cozinheiro sabe fazer.}
\end{center}

\begin{center}
	\emph{Processos de Desenho que vão assegurar a \textbf{Garantia} são: Disponibilidade, Capacidade, Continuidade e Segurança.}
\end{center}

\subsection{SLA,OLA,Contratos de Apoio}

\textbf{Resumo SLA,OLA, Contratos}
\begin{itemize}
	\item Lida com Area de Negocio = SLA / ANS
	\item Lida com Area Interna = OLA / ANO
	\item Lida com Fornecededores Externos = Contratos de Apoio
\end{itemize}

\subsection{Gerência de Capacidade}

\begin{center}
	\emph{Cuidado!}
	\emph{Gestao de Demandas também é uma atividade que pertence à Gerência de Capacidade.}
	\emph{Acontece Ger Demandas no Estágio de Desenho? Sim! Como uma atividade.}
\end{center}

\textbf{Subprocessos:}
\begin{itemize}
	\item Gerência de Capacidade de Negócios = Domínio Estratégico
	\item Gerência de Capacidade de Serviços = Domínio Tático
	\item Gerência de Capacidade de Componentes = Domínio Operacional
\end{itemize}

\subsection{Gerência de Disponibilidade}

\textbf{Indicadores}
\begin{itemize}
	\item MTRS = Restore = Downtime = Sustentabilidade
	\item MTBF = Between Failures = Uptime = Confiabilidade = Você confia na sua BF
	\item MTBSI = Between Service Incidents = UPTIME + DOWNTIME
	\item MTTR = To Repair => MTTR < MTRS
\end{itemize}

\begin{center}
	\emph{Está funcionando? Está UP? Você confia! Confiabilidade!}
	\emph{Não está funcionando? Está DOWN? Você Assusta! = Sustentabilidade!}
	\emph{Você confia na sua Best Friend = MTBF = Confiabilidade}
\end{center}

\subsubsection{Gerência de Continuidade de Serviços de TI}

\textbf{Opções de Recuperação = Resumo = Nome e MTRS (DOWNTIME)}
\begin{itemize}
	\item Gradual > Intermediária > Rápida > Imediata
	\item Dias ou Semanas > 1 a 3 dias > até 1 dia > Sem Interrupção
	\item Cold > Warm > Hot > Mirroring ou Split
\end{itemize}

\subsection{Ger. de Segurança da Informação}

	\begin{center}
		\emph{É responsabilidade da Ger. de Segurança fazer o monitoramento e gestão de \textbf{incidentes de segurança} , relatórios, revisões de incidentes de segurança;}
	\end{center}


\textbf{Resumo das Estratégias de resposta}
\begin{itemize}
	\item Prevenção = Evitar
	\item Redução = Minimizar
	\item Detecção = Descobrir
	\item Repressão = reprimir a continuação ou a repetição do incidente
	\item Correção = reparar o dano
\end{itemize}


\subsection{Gerência de Fornecedores (ainda é Desenho)}

\begin{center}
	\emph{Obter retorno adequado (value for money)}
\end{center}


\section{ITIL: Livro de Transição de Serviços}

\begin{center}
	\emph{SLR => Estágio de Transição => SLA}
\end{center}


\textbf{PROCESSOS DE TRANSIÇÃO}
\begin{itemize}
	\item Ger. de Configuração e Ativos de Serviço
	\item Ger. de Mudanças
	\item Ger. do Conhecimento
	\item Planejamento e Suporte da Transição
	\item Ger. de Liberação e Implantação
	\item Validação e Teste de Serviços
	\item Avaliação de Mudanças
\end{itemize}

	\begin{center}
		\emph{Os 3 primeiros Configuração, Mudança e Conhecimento são necessidades do Estágio de Transição, mas estão presentes em todo o Ciclo de Vida do serviço; Eles fazem as mesmas atividades o tempo todo;}
	\end{center}

\subsubsection{Gerência de Mudanças = CMS}

	\begin{center}
		\emph{Garantir que métodos padronizados sejam usados para o tratamento eficiente e tempestivo de todas as mudanças, que todas as mudanças sejam registradas no \textbf{CMS (Sistema de Gerência de Configuração)} e que o risco para o negócio seja otimizado;}
	\end{center}

\textbf{Tipos de Requisição de Mudança}
\begin{itemize}
	\item Mudança Padrão (standard) = Padronizada
	\item Mudança Emergencial = Tenho que avaliar impáctos
	\item Mudança Normal
\end{itemize}


\textbf{7 R's para Avaliar Mudança}
\begin{itemize}
	\item Requisitante
	\item Razão
	\item Retorno = Como volta?
	\item Riscos
	\item Recursos
	\item Responsável
	\item Relação
\end{itemize}


\subsection{Gerência de Configuração e Ativos}

\begin{center}
	\emph{Mantém informações sobre meus itens de configuração.}
\end{center}

\subsubsection{Categorias de CIs}

\textbf{Ciclo de Vida do Serviços = SDP \& Planos}
\begin{itemize}
	\item Pacotes de Desenho de Serviço (SDP)
	\item Plano de Mudanças
	\item Plano de Release
	\item Plano de Teste
	\item Planos de Ciclo de Vida
\end{itemize}

\textbf{Internos}
\begin{itemize}
	\item Artefatos de projeto
	\item Conjuntos que ajudam a entregar e manter serviços
	\item \emph{Para a FCC: Hardware, software, configurações lógicas de máquinas virtuais, rotas} 
\end{itemize}


\textbf{Externos}
\begin{itemize}
	\item Requerimento de Clientes
	\item Acordos com Clientes
	\item Releases de Fornecedores
	\item \emph{Para a FCC: Requerimentos e acordos com fornecedores} 
\end{itemize}


\textbf{Interface}
\begin{itemize}
	\item Itens que ajudam a entregar um serviço através de um provedor de interface
\end{itemize}

\textbf{Serviço}
\begin{itemize}
	\item Conjuntos de capacidade (Conhecimento, Processos, Gerenciamentos, Organizações)
	\item Conjuntos de recursos (Aplicações, Dados, Instalações e Capital)
	\item Pacotes de Serviços
	\item Pacotes de Liberação
	\item \emph{Para a FCC: Processos, relatórios do serviço, bases de conhecimento, pessoas, Modelo do Serviço, Pacote do Serviço, Pacote de Liberação} 
\end{itemize}


\textbf{Organização}
\begin{itemize}
	\item Características dos CIs
	\item Documentos relacionados a CIs
	\item Regulações
	\item Informações sobre produtos
	\item \emph{Para a FCC: Estratégia da empresa, políticas internas da empresa, leis, regulamentações} 
\end{itemize}

\subsection{Gerência de Ativos (Faz parte da Ger. de Cfg e Ativos)}

\begin{center}
	\emph{“Ativos Fixos” ou “Ativos Financeiros” = Ativos Patrimoniais}
\end{center}

\textbf{Escopo}
\begin{itemize}
	\item Software Asset Management (licenciamento) = Licenças e Mídias de Instalação
	\item Secure Libraries and Stores (armazenamento) = Equipamentos de Becape
	\item Definitive Spares (reposição)
	\item Definitive Media Library (linhas de base das releases)
\end{itemize}


\begin{center}
	\emph{Ger. de Configuração cuida das Informações}
	\emph{Ger. de Ativos cuida das coisas físicas que armazenam essas informações}
\end{center}


\textbf{Configuration Management System (CMS) ou Sistema de Gerência de Configuração}
\begin{itemize}
	\item Armazena informações sobre os CIs (Itens de Configuração)
\end{itemize}

\textbf{Definitive Media Library (DML)}
\begin{itemize}
	\item Biblioteca de Mídias
	\item Áreas de armazenamento de arquivos de armazenamento
	\item Áreas físicas: Hardware de Storage
\end{itemize}


\subsection{Gerência de Conhecimento = Konhecimento}

\begin{center}
	\emph{Principal produto do processo é o Sistema de Gestão do Konhecimento de Serviços (SKMS)}
\end{center}

\textbf{Camadas do SKMS}
\begin{itemize}
	\item Apresentação = Diversas visões
	\item Processamento de Conhecimento
	\item Integração de Informação
	\item Dados = CMDBs, DMLs
\end{itemize}


\subsection{Planejamento e Suporte da Transição}

\begin{center}
	\emph{Vai tratar dos aspectos organizacionais da transição.}
\end{center}


\textbf{Planejamento e Suporte da Transição}
\begin{itemize}
	\item Planejar
	\item Cuidar de Falhas
\end{itemize}

\subsection{Ger. de Liberação e Implantação = Release e Deployment}

\begin{center}
	\emph{É a parte técnica da transição.}
\end{center}


\textbf{Estratégias de liberação}
\begin{itemize}
	\item Big Bang = de uma vez
	\item Faseada = em partes
	\item Abordagem Push = Empurrar = Serviço é implantado enfiado
	\item Abordagem Pull = Puxar = Serviço disponível para implantação
	\item Automatizada
	\item Manual
\end{itemize}

\subsection{Validação e Teste de Serviços}

\begin{center}
	\emph{Prover evidência objetiva de que o serviço novo ou alterado suporta os requisitos de negócio, incluindo os SLA’s estabelecidos;}
\end{center}

\subsection{Avaliação de Mudanças}
	\begin{center}
		\emph{Garantir que o serviço continue sendo relevante, pelo estabelecimento de \textbf{métricas e métodos de mensuração apropriados};}
	\end{center}


% ###########################################

\section{ITIL: Livro Operação de Serviços}

\textbf{OPERAÇÃO}
\begin{itemize}
	\item Ger. de Eventos
	\item Ger. de Incidentes
	\item Ger. de Problemas
	\item Cumprimento de Requisições
	\item Ger. de Acesso
	\item Funções: Service Desk, Ger. Técnica,
	\item Ger. de Aplicações, Ger. de Operações
\end{itemize}

\textbf{MELHORIA CONTÍNUA}
\begin{itemize}
	\item Melhoria em 7 passos
\end{itemize}

\begin{center}
	\emph{No Estágio de Operações ocorre a efetiva entrega de valor ao Cliente.}
\end{center}

\textbf{1 - Manter a coisa funcionando: Funções}
\begin{itemize}
	\item Funções: Service Desk, Ger. Técnica, Ger. de Aplicações, Ger. de Operações
\end{itemize}


\textbf{2 - Resolver exceções e atender as solicitações que chegam: Processos}
\begin{itemize}
	\item Ger. de Eventos
	\item Ger. de Incidentes
	\item Ger. de Problemas
	\item Cumprimento de Requisições
	\item Ger. de Acesso
\end{itemize}

\textbf{Operações é o único estágio em que os serviços efetivamente entregam valor ao cliente.}

\subsection{Objetivos conflitantes}

\textbf{Visão interna (TI) x Visão externa (negócio)}
\begin{itemize}
	\item A visão técnica é necessária para gestão dos componentes dos serviços, mas não pode se sobrepor aos requisitos de qualidade dos usuários para desses serviços
\end{itemize}

\textbf{Estabilidade x Tempo de atendimento}
\begin{itemize}
	\item A infra-estrutura de TI deve ser estável para oferecer a disponibilidade esperada, ao mesmo tempo em que deve ser flexível para adaptar-se a mudanças de requisitos do negócio
\end{itemize}

\textbf{Qualidade do serviço x Custo do serviço}
\begin{itemize}
	\item Os serviços devem atender aos SLAs estabelecidos, ao menor custo possível e com uso otimizado dos recursos
\end{itemize}

\textbf{Atividades reativas x Atividades proativas}
\begin{itemize}
	\item É importante antecipar-se aos possíveis problemas, desde que isso não implique mudanças excessivas ou perda da capacidade de reação
\end{itemize}

\subsection{Conceitos básicos}

\begin{figure}[ht]\centering
	\includegraphics[width=\linewidth]{\figpath{20170725_214201.jpg}}
	\caption{Incidente Requisição}
	\label{fig:20170725_214201}
\end{figure}
%a figura \ref{fig:20170725_214201}

\textbf{Incidente = EFEITO}
\begin{itemize}
	\item Falha de um item de configuração
	\item Interrupção não planejada ou
	\item Redução na qualidade de um serviço de TI
\end{itemize}

	\begin{center}
		\emph{Incidente é o EFEITO}
	\end{center}

\textbf{Problema = CAUSA DESCONHECIDA}
\begin{itemize}
	\item Causa desconhecida de um ou mais incidentes, a ser investigada pelo gerenciamento de problemas
\end{itemize}

	\begin{center}
		\emph{Problema é a causa desconhecida do incidente}
	\end{center}

\textbf{Erro conhecido}
\begin{itemize}
	\item Problema cuja causa raiz é conhecida e para o qual existe (se possível) uma solução de contorno documentada
\end{itemize}

\textbf{Erro conhecido é o conjunto Resposta + Solução que pode ser:}
\begin{itemize}
	\item Solução de Contorno => Atua no Efeito (IMEDIATO) => Resolve o Incidente sem saber a causa = Desligar/Ligar o MODEM
	\item Solução Definitiva/Estrutural => Atua na Causa => Resolve o Problema e assim esse problema não gera novos incidentes;
\end{itemize}

\subsection{Operação de Serviços - Funções}


\textbf{Funções:}
\begin{itemize}
	\item Service Desk,
	\item Ger. Técnica,
	\item Ger. de Aplicações,
	\item Ger. de Operações
\end{itemize}

\subsection{Service Desk}

Fazer

\subsection{Gerência de Eventos}

Fazer

\subsection{Gerência de Incidentes = Atua no Efeito}

Gerencia de Incidentes atua no efeito. 

Faz diagnostico? Sim => Do Incidente (do efeito)

\textbf{Gerência de Incidentes - Atividades}
\begin{itemize}
	\item Identificação de Incidentes
	\item Registro de Incidentes
	\item Categorização de Incidentes
	\item Priorização de Incidentes
	\item Diagnóstico inicial
	\item Escalação de Incidentes
	\item Investigação e Diagnóstico
	\item Resolução e Recuperação
	\item Encerramento de Incidentes
\end{itemize}

\subsubsection{Priorização de Incidentes}

Verificar a figura \ref{fig:SS_2208_BW_3} e atentar para os códigos e respectivos tempos de resolução do alvo.

\begin{figure}[ht]\centering
	\includegraphics[width=\linewidth]{\figpath{SS_2208_BW_3.jpg}}
	\caption{Priorização de Incidentes}
	\label{fig:SS_2208_BW_3}
\end{figure}



\subsection{Gerência de Problemas = Resolver a causa raiz}

\textbf{Envolve aspectos}
\begin{itemize}
	\item Reativos
\begin{itemize}
	\item Solução de problemas em resposta a um ou mais Incidentes;
\end{itemize}
	\item Proativos
\begin{itemize}
	\item Identificação de causas recorrentes e suas soluções Estruturais;
\end{itemize}
\end{itemize}

\section{ITIL: Livro de Melhoria Contínua => Livro ignorado pelas bancas}


\begin{center}
	\emph{Objetivo:}
	\emph{Manter o valor para os clientes por meio da avaliação e melhoria contínua da}
	\emph{Qualidade dos serviços}
	\emph{Maturidade dos processos de gerenciamento}
\end{center}

\textbf{Combinação de princípios, práticas e métodos da}
\begin{itemize}
	\item Gestão de qualidade,
	\item Gestão de mudanças e
	\item Melhoria de capacidade
\end{itemize}


\textbf{Modelo de Melhoria Contínua: Como manter o momentum andando?}
\begin{itemize}
	\item Qual é a visão? <==> Visão de negócio, missão, metas e objetivos
	\item Onde estamos agora? <==> Objetivos de linha base
	\item Onde queremos estar? <==> Alvos mensuráveis
	\item Como chegamos lá? <==> Melhoria de serviços e processos
	\item Chegamos lá? <==> Medidas e Métricas
\end{itemize}

\subsubsection{Melhoria em 7 passos}

\begin{center}
	\emph{Esses 7 passos vão reaparecer no COBIT5, no Capítulo 7, como perguntas no Ciclo de Vida da Implementação. Na Pizza!}
\end{center}


\textbf{Quadrantes}
\begin{itemize}
	\item IV = x < 0 e y > 0 = Wisdom
	\item I = x > 0 e y > 0  = Data
	\item II = x > 0 e y < 0 = Information
	\item III = x < 0 e y < 0 = Knowledge
\end{itemize}

\subsubsection{Resumo dos 7 Passos da Melhoria em 7 passos}

\textbf{7 Passos}
\begin{itemize}
	\item Passo 1: Identificar a estratégia para melhoria
	\item Passo 2: Definir o que você quer medir?
	\item Passo 3: Obter os dados!
	\item Passo 4: Processar os Dados
	\item Passo 5: Analisar a informação e os dados
	\item Passo 6: Apresentar a informação
	\item Passo 7: Implementar a Melhoria
\end{itemize}

\subsubsection{7 Passos da Melhoria em 7 passos e seus itens:}

\textbf{Passo 1: Identificar a estratégia para melhoria}
\begin{itemize}
	\item Visão
	\item Necessidades do negócio
	\item Estratégia
	\item Metas Táticas
	\item Metas Operacionais
\end{itemize}

\textbf{Passo 2: Definir o que você quer medir?}

\textbf{Passo 3: Obter os dados!}
\begin{itemize}
	\item Quem? Como? Quando?
	\item Critérios para avaliar a integridade dos dados
	\item Metas operacionais
	\item Medidas de serviço
\end{itemize}

\textbf{Passo 4: Processar os Dados}
\begin{itemize}
	\item Frequência?
	\item Formatos?
	\item Ferramentas e Sistemas?
	\item Precisão?
\end{itemize}

\textbf{Passo 5:  Analisar a informação e os dados}
\begin{itemize}
	\item Marcas?(Trends?)
	\item Alvos?
	\item Melhorias requeridas?
\end{itemize}

\textbf{Passo 6: Apresentar a informação}
\begin{itemize}
	\item Sumário de assess (medições)
	\item Planos de Ação
	\item Etc...
\end{itemize}

\textbf{Passo 7: Implementar a Melhoria}


