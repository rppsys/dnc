
\section{criarDBTree}

O arquivo criarDBTree.py cria o banco de dados com os passos mais fundamentais.

Destalhes do banco ficam para depois.


\section{Passos Fundamentais}

\subsection{Modificadores de Passos}


Os modificadores de translação e rotação são declarados dentro de parenteses.

Podem ser:

\begin{itemize}
	\item \textbf{rot} - Modificador de Rotação
	\item \textbf{tra} - Modificador de Translação
\end{itemize}

\subsection{Modificadores de Rotação rot}


O modificador de rotação leva duas letras. A primeira indica o sentido e a segunda o giro.

\subsubsection{Primeira Letra - Sentido}

Sentidos:

\begin{itemize}
	\item \textbf{f} - Sentido para fora.
	\item \textbf{d} - Sentido para dentro.
\end{itemize}


Ver as figuras com as definições do que é para fora e o que é para dentro. 

Macete: Quando o braço abraça levando a palma da mão com os polegares para cima em direção ao peito o sentido é dentro. Sempre. Mesmo de costas.

Quando o braço se movimenta na direção da palma da mão com o polegares para baixo em direção a bater nas costas é fora. Sempre.


\subsubsection{Segunda Letra - Ângulo de rotação}

Rotações

\begin{itemize}
	\item \textbf{q} - 90 graus
	\item \textbf{v} - 180 graus
	\item \textbf{t} - 30 graus
\end{itemize}


\subsection{Exemplo de Modificadores de Rotação}


pD.FRT(rot=fq)

Leva o pé direito para frente rotacionando para fora (sentido negativo) em 90 graus.

pD.FRT(rot=dq)

Leva o pé direito para frente rotacionando para dentro (sentido positivo) em 90 graus.


Colocar aqui todas as possibilidades e estuda-las.


Fazer uma tabela Ver projetoDNC.ods




\subsection{Modificadores de Translação Tra}


A translação pode ocorrer a partir de 8 pontos a partir da posição final do pé.


Então vc verifica qual seria a posição final do pé e a partir dali traça uma estrela de 8 pontas. Norte Sul Leste Oeste e ai teremos as possibilidades.


Faço mais depois...


















